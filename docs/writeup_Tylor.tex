\documentclass{article}
\title{Strategic Wi-Fi Maintainer}
\author{
  Tylor Reeves\\
  \and
  Ruslan Nabioullin
}

\begin{document}
\maketitle

\section{Introduction}
Over the past three decades, mobile computing has proliferated and come to
depend heavily upon the Internet. Unlike desktop computers, mobile computers,
which include portable computers (laptops, netbooks, etc.), mobile phones, and
wearable computers (e.g., smartwatches), require a wireless interface for
network connectivity. This is achieved mainly by the use of more modern cellular
and wireless local area network technologies, the latter typically implemented
as Wi-Fi. Although cellular technology offers relatively large ranges, it
suffers from a number of relative disadvantages including, but not limited to
the following, the fact that many portable computers lack a cellular modem, a
cellular modem might be incompatible with a service provider, esp. outside of a
region or nation, cellular service presents privacy concerns, a user may simply
lack a data plan, and service may be of low quality or even unavailable in some
areas. Wi-Fi tends to be relatively faster and cheaper, in many cases free of
charge, and in any case adds redundancy. Therefore, Wi-Fi is an important
complementary technology.

However, existing Wi-Fi connection managers are usually intended to be used only
with preconfigured networks; not all can connect automatically to new networks,
and for those that can, they might not do so in an intelligent and aggressive
manner. This places an unnecessary burden upon the user, who must manually scan
for networks, connect to promising network(s), and test them. The goal of our
project is to design a user-friendly proof of concept utility that will automate
this by attempting to automatically maintain a working Wi-Fi connection at all
costs in a strategic and efficient manner.

\section{Approach}
The utility will be written entirely in Racket and will require GNU/Linux with a
compatible Wi-Fi network interface controller. Being a proof of concept program,
it will have a minimal interface consisting of simple status messages printed
out to the command line interface.

The utility will constantly check for Internet connectivity by means of the ping
utility. If connectivity is lost, the utility will proceed to attempt to
establish a connection from available Wi-Fi networks.

First, the utility will scan for all Wi-Fi networks in range by means of the
iwlist utility and utilize an algorithm to determine the order in which to
attempt establishing a connection so as to: 1. minimize the time spent
attempting to connect; 2. maximize the probability of the connection being of
high quality; 3. maximize propriety and minimize potential legal liability for
the user. The algorithm will calculate a score for every network. A priori
criteria will include signal strength, network name, and security in use, if
any. Additionally, it will consult an internal database of past networks
utilized to possibly add the criterion of the presence of connectivity in the
past. Next, it will attempt to connect to the networks in order of descending
score by means of the iwconfig and dhclient utilities. If security is employed,
the utility will attempt to crack it by means of Aircrack-ng or a similar
utility if not done so in the past. Once a connection to the network is
established, the utility will test Internet connectivity by means of the ping
utility. If the connection is suitable, the utility will revert back to its
initial state of ensuring that there is connectivity. Else, for example if a
login is required or the network is otherwise nonfunctional, the utility will
restart the process with the network of the next lower score, if any.  If no
untested networks remain, the utility will continue to search for a viable
network indefinitely, either until it finds one or the user terminates the
utility.

Functions will be developed which will execute the abovementioned utilities by
means of Racket's process functions. Output from them will be subsequently
parsed by means of Racket's byte and string input functions and pertinent data
returned to the caller. The utility will update its database when it determines
the connectivity status and/or security credentials, if any, of a network, and
maintain the database by periodically expunging any stale entries.

\section{Demonstration and Evaluation}
The project will be demonstrated in a live and realistic manner. A netbook will
be set up displaying the utility and an indication of the presence of simulated
Internet connectivity to the audience. Several Wi-Fi networks will be physically
set up with varying characteristics so as to simulate real-life environments and
to demonstrate key functionality; e.g., the environment may consist of two
insecure networks of different signal strength, a secure network and an insecure
network lacking simulated Internet connectivity. Physical movement of the user
will be simulated by the disabling of certain Wi-Fi networks.

The project shall be deemed minimally successful if the utility is able to
maintain a functioning Internet connection to the extent possible from insecure
Wi-Fi networks in a timely manner, without user intervention. Ideally, the
utility will also be able to gain access and connect to secured Wi-Fi networks
as well, purely as a proof of concept.

\section{Deliverables Schedule}
\begin{tabular}[t]{|l|p{.15\linewidth}|p{.3\linewidth}|p{.3\linewidth}|}
  \hline
  Date & Description & What I will do & What my partner will do\\\hline
  11-24 & First deliverable & Development of the scoring algorithm. &
  Development of an interface to the scanning of Wi-Fi networks.\\\hline
  12-1 & Second deliverable & Development of the database. & Development of
  remaining basic interfaces.\\\hline
  12-8 & Final project & Testing and debugging. & Development of a cracking
  interface (optional). Testing and debugging.\\\hline
\end{tabular}
\end{document}